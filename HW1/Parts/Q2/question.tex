% !TeX root = ../../main.tex
\documentclass[../AAE590ACA_HW1.tex]{subfiles}
\begin{document}
\section*{(1.2) Three-Body Problem}
\rhead{Problem 2}

Next, we consider the spacecraft orbital motion in a three-body system.

\subsection*{(a) Circular Restricted Three-Body Problem (CR3BP)}

Consider the Earth-Moon system using the circular restricted three-body problem (CR3BP). The parameter values for this problem are provided in Table \ref{table:dynparam2}.

\begin{table}[h]
    \centering
    \caption{Assumed dynamical parameter values (Problem (1.2))}
    \begin{tabular}{lllc}
        \hline \hline 
        Parameter & Symbol & Value & Unit \\
        \hline  
        Earth-Moon distance & \( d_{\text{Earth-Moon}} \) & \( 3.8475 \times 10^{5} \) & km \\
        Earth-Moon barycenter GM & \( G(M_{\text{Earth}} + M_{\text{Moon}}) \) & \( 4.0350 \times 10^{5} \) & km\(^3\)/s\(^2\) \\
        Mass ratio & \( \mu \) & \( 1.2151 \times 10^{-2} \) & - \\
        \hline \hline 
    \end{tabular}
    \label{table:dynparam2}
\end{table}

Simulate the spacecraft's orbital motion in the CR3BP system with the initial conditions and propagation times given in Table \ref{table:synIC}. Plot each trajectory in the synodic frame in the dimensional system, marking the positions of the Earth and Moon.

\begin{table}[h]
    \centering
    \caption{Initial conditions for CR3BP simulations (non-dimensional)}
    \begin{tabular}{cccccccc}
        \hline \hline 
        IC \# & \( x_{0} \) & \( y_{0} \) & \( z_{0} \) & \( \dot{x}_{0} \) & \( \dot{y}_{0} \) & \( \dot{z}_{0} \) & Propagation time \\
        \hline 
        IC-1 & 1.2 & 0 & 0 & 0 & -1.06110124 & 0 & 6.20628 \\
        IC-2 & 0.85 & 0 & 0.17546505 & 0 & 0.2628980369 & 0 & 2.5543991 \\
        IC-3 & 0.05 & -0.05 & 0 & 4.0 & 2.6 & 0 & 15.0 \\
        \hline \hline 
    \end{tabular}
    \label{table:synIC}
\end{table}

\subsection*{(b) Third-Body Perturbation in the ECI Frame}

The orbital motion under the influence of Earth and Moon gravity can also be modeled as a perturbed two-body problem, where the Moon exerts a third-body perturbation. To do this, re-define the ECI frame such that:
\begin{itemize}
\item \( \hat{\boldsymbol{n}}_{1} \) is aligned with the Earth-Moon line at the epoch.
\item \( \hat{\boldsymbol{n}}_{3} \) is aligned with the normal vector of the Earth-Moon orbital plane.
\end{itemize}

The Moon's third-body perturbation in the ECI frame is given by:

\[
\boldsymbol{a}_{\mathrm{moon}}=-\mu_{\mathrm{moon}}\left(\frac{\boldsymbol{r}-\boldsymbol{r}_{\mathrm{moon}}}{\left\|\boldsymbol{r}-\boldsymbol{r}_{\mathrm{moon}}\right\|_{2}^{3}}+\frac{\boldsymbol{r}_{\mathrm{moon}}}{\left\|\boldsymbol{r}_{\mathrm{moon}}\right\|_{2}^{3}}\right),
\]

where:
\begin{itemize}
\item \( \mu_{\text{moon}} = 4.9028 \times 10^{3} \) km\(^3\)/s\(^2\),
\item \( \boldsymbol{r}_{\text{moon}} \) is the Moon’s position relative to the ECI frame origin.
\end{itemize}

For simplicity, assume that the Moon's orbit is circular (similar to the CR3BP model), but about Earth instead of the barycenter.

Convert the initial conditions in Table 4 to position and velocity vectors in the ECI frame and propagate the system using the perturbed two-body equations under the Moon’s third-body perturbation. Show the results in both:
\begin{itemize}
\item The ECI frame.
\item The synodic frame.
\end{itemize}

\subsection*{(c) Consistency of Results}

Compare the results obtained in parts (a) and (b). Discuss the consistency between the two approaches. If discrepancies are observed, explain the possible causes.

\subsection*{(d) Extra Credit: Additional Perturbation}

For extra credit, choose another perturbing force of your choice and derive its explicit mathematical expression. Include it in the three-body simulation (either in the CR3BP or ECI frame, at your discretion). Run the simulation with the same initial conditions from Table 4 and compare the results to the previous cases.

The chosen perturbation does not need to be covered in class.
\end{document}